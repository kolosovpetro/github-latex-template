Your introduction here.
Include some references~\cite{derivative_of_polynomials_via_double_limit,
    alekseyev2018mathoverflow,
    oeis_coefficients_u_m_l_k_defined_by_polynomial_identity_3}.
Lorem Ipsum is simply dummy text of the printing and typesetting industry.
Lorem Ipsum has been the industry's standard dummy text ever since the 1500s, when an unknown printer took a galley
of type and scrambled it to make a type specimen book.
It has survived not only five centuries, but also the leap into electronic typesetting, remaining essentially unchanged.
It was popularised in the 1960s with the release of Letraset sheets containing Lorem Ipsum passages, and more
recently with desktop publishing software like Aldus PageMaker including versions of Lorem Ipsum.

Image example
\begin{figure}[H]
    \centering
    \includegraphics[width=1\textwidth]{sections/images/build_the_manuscript_rider}
    ~\caption{Image example.}\label{fig:figure}
\end{figure}

\input{sections/figures/05_fig_coefficients_a}

\begin{equation*}
    \llceil \nobarfrac{a}{b} \rrfloor_{m}
\end{equation*}
\begin{equation*}
    \llceilCoefficient{a}{b}{m}
\end{equation*}

And for any natural $m$ we have polynomial identity
\begin{equation}
    x^m = \sum_{k=1}^{m} T(m, k) \centralFactorial{x}{k}
    \label{eq:knuth-power-identity}
\end{equation}
where $\centralFactorial{x}{k}$ denotes central factorial defined by
\begin{equation*}
    \centralFactorial{x}{n} = x \fallingFactorial{x+\frac{n}{2}-1}{n-1}
\end{equation*}
where $\fallingFactorial{n}{k} = n (n-1) (n-2) \cdots (n-k+1)$ denotes falling factorial in Knuth's notation.
In particular,
\begin{equation}
    \label{eq:falling-factorial}
    \centralFactorial{x}{n}
    = x \left( x+\frac{n}{2}-1 \right) \left( x+\frac{n}{2}-1 \right) \cdots \left (x+\frac{n}{2}-n-1 \right)
    = x \prod_{k=1}^{n-1} \left( x+\frac{n}{2}-k \right)
\end{equation}
This is an equation reference~\eqref{eq:knuth-power-identity}.

Continuing similarly, we are able to derive the formula for multifold sums of powers,
which is
\begin{theorem}[Multifold sums of powers via Newton's series]
    \label{theorem:multifold-sums-of-powers-via-newtons-series}
    \begin{mdframed}
        For non-negative integers $r,n,m$ and an arbitrary integer $t$
        \begin{align*}
            \KnuthRFoldSum{r}{n}{m} = \sum_{j=0}^{m} \Delta^{j} t^{m} \left[ \left( \sum_{s=1}^{r} (-1)^{j+s-1} \binom{j+t-1}{j+s} \KnuthRFoldSum{r-s}{n}{0} \right) + \binom{n-t+r}{j+r} \right]
        \end{align*}
    \end{mdframed}
    \begin{proof}
        By Newton's series for power and repeated
        applications of the segmented hockey stick identity.
    \end{proof}
\end{theorem}
