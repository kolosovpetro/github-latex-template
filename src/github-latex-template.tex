\documentclass[12pt,letterpaper,oneside,reqno]{amsart}
\usepackage{amsmath}
\usepackage{amssymb}
\usepackage{amsthm}
\usepackage{float}
\usepackage{mathrsfs}
\usepackage{colonequals}
\usepackage[font=small,labelfont=bf]{caption}
\usepackage[unicode,pdfpagelabels,hyperindex,colorlinks=true,linkcolor=red,urlcolor=blue,citecolor=red]{hyperref}
\usepackage{graphicx}
\emergencystretch=1em
\usepackage{array}
\usepackage{enumitem}
\usepackage{etoolbox}
\usepackage{booktabs}
\usepackage{mdframed}

% margins and layout
\linespread{1.7}
\usepackage[left=1in,right=1in,bottom=1in,top=1in]{geometry}
\apptocmd{\sloppy}{\hbadness 10000\relax}{}{}
\raggedbottom

\newcommand \coeffA [3][A] {{\mathbf{#1}}_{#2,#3}}
\newcommand \polynomialP [4][P]{{\mathbf{#1}}\sp{#2} \sb{#3}(#4)}
\newcommand \bernoulli [2][B] {{#1}\sb{#2}}

%\newcommand \anglePower [2]{\langle #1 \rangle \sp{#2}}
%\newcommand \curvePower [2]{\{#1\}\sp{#2}}

% central factorials and related symbols
\newcommand \centralFactorial [2] {#1^{[#2]}}
\newcommand \fallingFactorial [2] {\left(#1 \right)^{\underline{#2}}}
\newcommand{\stirlingii}{\genfrac{\{}{\}}{0pt}{}}
\newcommand{\eulerian}[2]{\genfrac{\langle}{\rangle}{0pt}{}{#1}{#2}} % Eulerian

\newcommand{\KnuthRFoldSum}[3]{\Sigma^{#1}\,{#2}^{#3}}

% for llceil coeffcient
\newcommand{\nobarfrac}{\genfrac{}{}{0pt}{}}
\def\llceil{\left\lceil\kern-3.5pt\left\lceil}
\def\rrfloor{\right\rfloor\kern-3.5pt\right\rfloor}
\newcommand \llceilCoefficient [3] {\llceil \nobarfrac{#1}{#2} \rrfloor_{#3}}

% ~~~ Rascal numbers ~~~
%\newcommand \rascalNumber [3] {\binom{#1}{#2}_{#3}}
%\newcommand \north[0] {\mathbf{North}}
%\newcommand \south[0] {\mathbf{South}}
%\newcommand \west[0] {\mathbf{West}}
%\newcommand \east[0] {\mathbf{East}}

% ~~~~ 1-q pascal notation~~~~
%\newcommand{\genstirlingI}[3]{%
%    \genfrac{[}{]}{0pt}{#1}{#2}{#3}%
%}
%\newcommand{\genstirlingII}[3]{%
%    \genfrac{\{}{\}}{0pt}{#1}{#2}{#3}%
%}
%\newcommand{\oneQBinomial}[3]{\genstirlingI{}{#1}{#2}^{#3}}

% free foot note
\let\svthefootnote\thefootnote
\newcommand\freefootnote[1]{%
    \let\thefootnote\relax%
    \footnotetext{#1}%
    \let\thefootnote\svthefootnote%
}


\newtheorem{theorem}{Theorem}[section]
\newtheorem{corollary}[theorem]{Corollary}
\newtheorem{proposition}[theorem]{Proposition}
\newtheorem{observation}[theorem]{Observation}
\newtheorem{lemma}[theorem]{Lemma}
\newtheorem{claim}[theorem]{Claim}
\newtheorem{example}[theorem]{Example}
\newtheorem{conjecture}[theorem]{Conjecture}
\newtheorem{definition}[theorem]{Definition}
\newtheorem{question}[theorem]{Question}
\newtheorem{remark}[theorem]{Remark}
\newtheorem{assumption}[theorem]{Assumption}

%\numberwithin{equation}{section}

\title[LaTeX Template for GitHub]
{LaTeX Template for GitHub}
\author[Petro Kolosov]{Petro Kolosov}
\date{\today}

% metadata
% 05A19 — Combinatorial identities
% 05A10 — Factorials, binomial coefficients, combinatorial functions
% 41A15 — Spline approximation
% 11B68 — Bernoulli and Euler numbers and polynomials
% 11B73 — Bell and Stirling numbers
% 11B83 — Special sequences and polynomials
\subjclass[2010]{05A19, 05A10, 41A15, 11B68, 11B73, 11B83}
\input{metadata/keywords.tex}
\hypersetup{
    pdftitle={LaTeX Template for GitHub},
    pdfproducer={LaTeX},
    pdfcreator={pdflatex},
    pdfauthor={Petro Kolosov},
    pdfsubject={LaTeX Template for GitHub},
    pdfkeywords={Sums of powers, Power sums, Polynomials, Power function, Polynomial identities, Newton's interpolation formula, Finite differences, Interpolation, Approximation, Binomial coefficients, Multinomial coefficients, Binomial theorem, Faulhaber's formula, Bernoulli numbers, Bernoulli polynomials, Derivatives, Differential calculus, Partial differential equations, Discrete convolution, Sums of convolved powers Convolved power sums, Combinatorics, Central factorial numbers, Stirling numbers, Eulerian numbers, Worpitzky identity, Pascal's triangle, Dynamic systems, Time scales, OEIS}
}
\address{DevOps Engineer}
\email{kolosovp94@gmail.com}
\urladdr{https://kolosovpetro.github.io}

\begin{document}

    \begin{abstract}
        \input{sections/abstract}
    \end{abstract}

    \maketitle

    \freefootnote{DOI: \url{https://doi.org/10.5281/zenodo.18040979}}

    \tableofcontents


    \section{Introduction} \label{sec:introduction}
    Your introduction here.
Include some references~\cite{derivative_of_polynomials_via_double_limit,
    alekseyev2018mathoverflow,
    oeis_coefficients_u_m_l_k_defined_by_polynomial_identity_3}.
Lorem Ipsum is simply dummy text of the printing and typesetting industry.
Lorem Ipsum has been the industry's standard dummy text ever since the 1500s, when an unknown printer took a galley
of type and scrambled it to make a type specimen book.
It has survived not only five centuries, but also the leap into electronic typesetting, remaining essentially unchanged.
It was popularised in the 1960s with the release of Letraset sheets containing Lorem Ipsum passages, and more
recently with desktop publishing software like Aldus PageMaker including versions of Lorem Ipsum.

Image example
\begin{figure}[H]
    \centering
    \includegraphics[width=1\textwidth]{sections/images/build_the_manuscript_rider}
    ~\caption{Image example.}\label{fig:figure}
\end{figure}

\input{sections/figures/05_fig_coefficients_a}

\begin{equation*}
    \llceil \nobarfrac{a}{b} \rrfloor_{m}
\end{equation*}
\begin{equation*}
    \llceilCoefficient{a}{b}{m}
\end{equation*}

And for any natural $m$ we have polynomial identity
\begin{equation}
    x^m = \sum_{k=1}^{m} T(m, k) \centralFactorial{x}{k}
    \label{eq:knuth-power-identity}
\end{equation}
where $\centralFactorial{x}{k}$ denotes central factorial defined by
\begin{equation*}
    \centralFactorial{x}{n} = x \fallingFactorial{x+\frac{n}{2}-1}{n-1}
\end{equation*}
where $\fallingFactorial{n}{k} = n (n-1) (n-2) \cdots (n-k+1)$ denotes falling factorial in Knuth's notation.
In particular,
\begin{equation}
    \label{eq:falling-factorial}
    \centralFactorial{x}{n}
    = x \left( x+\frac{n}{2}-1 \right) \left( x+\frac{n}{2}-1 \right) \cdots \left (x+\frac{n}{2}-n-1 \right)
    = x \prod_{k=1}^{n-1} \left( x+\frac{n}{2}-k \right)
\end{equation}
This is an equation reference~\eqref{eq:knuth-power-identity}.

Continuing similarly, we are able to derive the formula for multifold sums of powers,
which is
\begin{theorem}[Multifold sums of powers via Newton's series]
    \label{theorem:multifold-sums-of-powers-via-newtons-series}
    \begin{mdframed}
        For non-negative integers $r,n,m$ and an arbitrary integer $t$
        \begin{align*}
            \KnuthRFoldSum{r}{n}{m} = \sum_{j=0}^{m} \Delta^{j} t^{m} \left[ \left( \sum_{s=1}^{r} (-1)^{j+s-1} \binom{j+t-1}{j+s} \KnuthRFoldSum{r-s}{n}{0} \right) + \binom{n-t+r}{j+r} \right]
        \end{align*}
    \end{mdframed}
    \begin{proof}
        By Newton's series for power and repeated
        applications of the segmented hockey stick identity.
    \end{proof}
\end{theorem}



    \section*{Conclusions}
    \input{sections/conclusions}


    \section*{Acknowledgements}
    The author is grateful to [Full Name] for his valuable contribution [contribution]
about the fact that [interesting claim].


    \bibliographystyle{unsrt}
    \bibliography{github-latex-template}

    \clearpage

    \noindent
\\[1em]
\textbf{Version:}
\texttt{Local-0.1.0}
\\[0.2em]
\noindent
\textbf{License:}
This work is licensed under a \href{https://creativecommons.org/licenses/by/4.0/}{\texttt{CC BY 4.0 License}}.
\\[0.2em]
\noindent
\textbf{Sources}:
\href{https://github.com/kolosovpetro/github-latex-template}{\texttt{github.com/kolosovpetro/github-latex-template}}
\\[0.2em]
\noindent
\textbf{ORCID}:
\href{https://orcid.org/0000-0002-6544-8880}{\texttt{0000-0002-6544-8880}}
\\[0.2em]
\noindent
\textbf{Email}:
\href{mailto:kolosovp94@gmail.com}{\texttt{kolosovp94@gmail.com}}


\end{document}
